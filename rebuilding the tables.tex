\documentclass{article}

% Language setting
% Replace `english' with e.g. `spanish' to change the document language
\usepackage[english]{babel}

% Set page size and margins
% Replace `letterpaper' with`a4paper' for UK/EU standard size
\usepackage[letterpaper,top=2cm,bottom=2cm,left=3cm,right=3cm,marginparwidth=1.75cm]{geometry}

% Useful packages
\usepackage{amsmath}
\usepackage{graphicx}
\usepackage{algpseudocode}
\usepackage[colorlinks=true, allcolors=blue]{hyperref}

\title{Rebuilding the tables}
\author{Yehuda Michelson}
\usepackage[mathletters]{ucs}
\DeclareUnicodeCharacter{955}{$\lambda$}
\begin{document}
\maketitle

\section{Introduction}
In this document, I intend to describe the code I wrote, It's purpose is to rebuild a shuffled array of balls, which is part of our implementation of the oblivious RAM.
\subsection{Overview}
The code is composed of three stages:
\begin{enumerate}
    \item Balls into bins.
    \item Select \& move the secret load.
    \item Tight compaction on the secret Load
\end{enumerate}
\subsection{inputs and configurations}
\begin{itemize}
    \item Input: $DATA\_LOCATION$ - The path to the original balls.
    \item $N$ - The amount of balls.
    \item $BALL\_SIZE$ - The size of a ball in bytes.
    \item $\mu$ - The average amount of balls per bin.
    \item $NUMBER\_OF\_BINS$ (also referred to as $B$) - the amount of bins $N/\mu$.
    \item $BIN\_SIZE$ - the max size of a bin $2\mu$
    \item $\varepsilon$ - The fraction of balls to be moved to the overflow pile $\frac{1}{\log{\lambda}}$.
\end{itemize}
\section{Balls into bins}
In this stage the code goes over the balls from the original array. 
\begin{itemize}
    \item $2\mu$ balls at a time, to reduce round trips. this is done because our code is assumed to have approximately $2\mu$ size of local memory    
    \item The code determines to which bin each ball is thrown using a $AES$ pseudo random function.
    \item The first ball in each bin contains the capacity of the bin (meaning how much of the bin is currently full) and is updated with every bulk insertion.
\end{itemize}

\section{Select \& move the secret load}
In this stage the code goes over the bins from the previous stage.
\begin{itemize}
    \item It goes over $1/\varepsilon$ bins at a time, to reduce round trips. This is done because our code is assumed to have approximately $2\mu$ size of local memory, and from each bin we read $2\mu\varepsilon$ as explained in the next item.
    \item from each bin - we read the capacity-ball and then we read the $2\mu\varepsilon$ top most balls of the bin. This is done because as explained in a previous document - the secret load would be there with high likelihood.
    \item For each bin, the code selects a secret threshold and copies only the balls above the threshold.
    \item from the $1/\varepsilon$ bins, the code takes the copied balls, and adds up enough dummies at the end so that there are $2\mu$ balls. Then the code writes the already tightly compacted bin to the overflow pile.
\end{itemize}
\section{Tight compaction on the secret Load}
Currently we have in the overflow pile $\varepsilon B$ bins which are already tightly compacted. We assume a mixed stripe along the middle and do a tight compaction on it (the likelihood of this mixed stripe still needs to be proven)


\begin{center}
\begin{tabular}{||c c c c||} 
 \hline
 $n$ & $LM$ & $\Pr[X\leq \mu - 0.5\varepsilon\mu]$ & $\Pr[Y\geq \mu - 0.5\varepsilon\mu]$ \\ [0.5ex] 
 \hline\hline
 $2^{30}$ & $\sim4*10^{-4}$ & $\sim4*2^{-50}$ & $\sim2*2^{-50}$\\ 
 \hline
 $2^{25}$ & $\sim7*10^{-3}$ & $\sim6*2^{-43}$ & $\sim3*2^{-43}$\\ 
 \hline
  $2^{20}$ & $\sim0.122$ & $\sim3*2^{-37}$ & $\sim2^{-37}$\\ 
 \hline
\end{tabular}
\end{center}
\subsection{Results}
\begin{itemize}
    \item $N = 2^{20}$ 
    \item $BALL\_SIZE = 16$ bytes
    \item $\mu = 30\log^3{\lambda} = 21870$
    \item $NUMBER\_OF\_BINS = 47$
    \item $\varepsilon = \frac{1}{9}$
\end{itemize}

\begin{center}
\begin{tabular}{||c c c c c||} 
 \hline
 stage & RT-write & RT-read & Ball-write & Ball-read \\ [0.5ex] 
 \hline\hline
 balls into bins& 24 & 48 & 1050888 & 1050888 \\ 
 \hline
 move secret load & 6 & 12 & 262440 & 262494 \\ 
 \hline
  tight compaction & 3 & 3 & 131220 & 131220 \\ 
  [0.5ex] 
 \hline\hline
 overall & 33 & 63 & 1444548 & 1444602 \\ \hline 
\end{tabular}
\end{center}
\end{document}